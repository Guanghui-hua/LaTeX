\documentclass[a4paper,12pt]{ctexart}     %页面大小和字体大小

\usepackage{ctex}
\usepackage{mathptmx}
\usepackage{amsmath}

\usepackage{url}  % 添加网页

\usepackage{listings}  %添加代码

\usepackage{graphicx} %插入图片的宏包
\usepackage{float} %设置图片浮动位置的宏包
\usepackage{subfigure} %插入多图时用子图显示的宏包

%for long table
\usepackage{longtable}
\usepackage{booktabs}  % 三线表


\usepackage{geometry}
\geometry{left=3.17cm, right=3.17cm, top=2.54cm, bottom=2.54cm}   %页边距

\linespread{1.25}      %设置行距


% 摘要设置  -->设置为4号字
\renewcommand{\abstractname}{\textbf{\songti\zihao{4}摘\quad 要}}
\usepackage{setspace}  %行间距的宏包



% 设置标题
\ctexset{
	section={
		format=\bfseries\zihao{4}\songti
	},
	subsection={
		format=\bfseries\zihao{-4}\songti
	}
}

% 设置目录
\setcounter{tocdepth}{2} %设定目录深度为2,即只显示到二级标题为止

% 设置页码
\usepackage{fancyhdr}
\setcounter{page}{1}  %从第一页开始
\usepackage{lastpage} 
\pagestyle{fancy} % 选用 fancy style % 其余同 plain style
\fancyhf{} % 清空当前设置
%\fancyhead{} %清空所有页眉
\renewcommand{\headrulewidth}{0pt}  %页眉线宽,设为0可以去页眉线
\cfoot{} 
\rfoot{\textbf{\thepage/\pageref{LastPage}}}

% 设置附录
\renewcommand\appendix{\par
	\setcounter{section}{0}
	\setcounter{subsection}{0}
	\gdef\thesection{附录 }%\Alph{section}}
	\gdef\thesubsection{附录\arabic{subsection}.}}


% 设置 表的编号的字体大小	
\usepackage{caption}
\captionsetup{font={small,stretch=1.25},justification=raggedright}


\usepackage{listings}  %添加代码
% 设置代码格式
\RequirePackage{color,xcolor}
% 设置代码的默认样式
\lstset{
	frame=none,% 取消边框
	breaklines=true,% 允许自动断行
	% breakatwhitespace=true,% 使用此命令仅允许在空格处自动断行
	showstringspaces=false,% 不显示字符串中的空格
	basicstyle=\small\ttfamily,% 设置代码基本样式
	flexiblecolumns=true,% 改善字母间距
	keywordstyle=\color{blue},% 设置关键词样式
	stringstyle=\color[rgb]{0.75,0,0.75},% 设置字符串样式
	commentstyle=\songti\color[rgb]{0,0.5,0},% 设置注释样式
	tabsize=4,% 设置制表符缩进
}

% 设置python代码环境
\lstnewenvironment{python}[1][]{
	\lstset{
		language=Python,
		keywordstyle=\color[RGB]{255,119,0},% 设置Keywords样式
		morekeywords={as},% 将特定单词加入Kewords中
		deletekeywords={print},%从 keywords中去除特定单词
		keywordstyle=[2]\color[RGB]{144,0,144},% 设置Builtins样式
		morekeywords=[2]{print},% 将特定单词加入Builtins中
		stringstyle=\color[RGB]{0,170,0},% 设置字符串样式
		commentstyle=\songti\color[RGB]{221,0,0},% 设置注释样式	
		#1
	}
}{}









\begin{document}
	
	本文采用李航老师《统计学习方法》一文中的例子,假设有4个盒子,每个盒子里面都装有红白两种颜色的球,盒子里面的红白球的数量有表\ref{tab:hezi}给出
	
	转移概率矩阵
	
	\begin{table}[htbp]\songti\zihao{5} 
		\begin{center}
			\renewcommand\arraystretch{2}         %表格内部 1.5 倍行距离
			\caption{模型数据 \label{tab:hezi}} 
			{\tabcolsep0.3in  %设置列间距
				
				\begin{tabular}{|c|c|c|c|c|}\hline   %开始表格环境,{|c|c|c|}表示文字居中的三列,\hline...\hline表述画两条并排的水平线。
					%\hline必须用于首行之前或者换行命令之后。
					
					\small 盒子编号&1&2&3&4\\\hline   %&是数据分割符号
					红球数&5&3&6&8\\\hline
					白球数&5&7&4&2\\\hline
				\end{tabular}
			}
		\end{center}
	\end{table}
	
	按照下面的方法抽球,产生一个球的颜色的观测序列,开始时以等概率从4个盒子里面选取一个盒子,然后从盒子里面摸一个球,然后再选取一个盒子,只不过下一个盒子并不是等概率选取,而是由于上一个盒子而有不同的概率被选中,其中具体转移概率参看表\ref{tab:zhuanyi},不同盒子中由于不同颜色的球的数量并不一样,所以抽到不同颜色的球的概率是不一样的。这就是一个简单的隐马尔可夫模型
	
	
	观测概率矩阵
	
	\begin{table}[htbp]\songti\zihao{5} 
		
		\begin{center}
			\renewcommand\arraystretch{2}         %表格内部 1.5 倍行距离
			\caption{观测概率 \label{tab:shuoming}} 
			{\tabcolsep0.3in
				
				\begin{tabular}{|c|c|c|}\hline   %开始表格环境,{|c|c|c|}表示文字居中的三列,\hline...\hline表述画两条并排的水平线。
					%\hline必须用于首行之前或者换行命令之后。
					
					\small &$ v_1 $&$ v_2$\\\hline   %&是数据分割符号
					$ q_1 $&0.5&0.5\\\hline
					$ q_2 $&0.3&0.7\\\hline
					$ q_3 $&0.6&0.4\\\hline
					$ q_4 $&0.8&0.2\\\hline
				\end{tabular}
			}
		\end{center}
	\end{table}
	
	
	
	转移概率矩阵
	\begin{center}
		\begin{longtable}[htbp]\songti\zihao{5} 
			
				\renewcommand\arraystretch{2}         %表格内部 1.5 倍行距离
				%\caption{Transform } \\
				\label{tab:zhuanyi}
				{\tabcolsep0.3in  %设置列间距
					
					\begin{tabular}{|c|c|c|c|c|}\hline   %开始表格环境,{|c|c|c|}表示文字居中的三列,\hline...\hline表述画两条并排的水平线。
						%\hline必须用于首行之前或者换行命令之后。
						
						\small &$ q_1 $ &$ q_2 $&$ q_3 $&$ q_4 $\\\hline   %&是数据分割符号
						$ q_1 $&0&1&0&0\\\hline
						$ q_2 $&0.4&0&0.6&0\\\hline
						$ q_3 $&0&0.4&0&0.6\\\hline
						$ q_4 $&0&0&0.5&0.5\\\hline
					\end{tabular}
				}
		\end{longtable}
	\end{center}



这里是一段话

这里是一段话


这里是一段话

这里是一段话

这里是一段话

这里是一段话

这里是一段话

这里是一段话

这里是一段话

这里是一段话
























	
	\begin{center}
		\begin{longtable}{cc} % 设置列间距
				\toprule    
				属性 & 符号  \\    
				\midrule 
				所有可能状态的个数 &$ N $ \\
				所有可能观测的个数 & $ M $ \\ 
				状态序列的个数 & $ T $  \\   
				单次状态 & $q_i$ \\ 
				所有可能状态的集合 & $ Q = \{ q_1,q_2,\dots,q_N\} $  \\
				单次观测 & $ v_i $  \\ 
				所有可能观测的集合 & $ V = \{ v_1,v_2,\dots,v_M\} $  \\
				状态序列 & $ I = \{i_1,i_2,\dots,i_T\} $  \\
				观测序列 & $ O = \{ o_1,o_2,\dots,o_T\} $ \\
				状态转移概率 & $ a_{ij} = P(i_{t+1} = q_j | i_t = q_i) $ \\
				条件观测概率 & $ b_j(k) = P(o_t = v_k| i_t = q_j)$ \\ 
				状态转移矩阵 & $ A = [a_{ij} ]_{N \times N} $ \\
				观测矩阵 & $ B = [b_j(k)]_{N \times M} $ \\
				初始状态概率 & $ \pi_i = P(i_1 = q_i)$ \\
				初始状态概率向量 & $ \pi = (\pi_i) $ \\
				所有参数 & $ \lambda = (A,B,\pi) $\\  
				\bottomrule   	
\end{longtable}

	\end{center}
\end{document}