\documentclass[a4paper]{ctexart}     %页面大小和字体大小

\usepackage{ctex}
\usepackage{mathptmx}
\usepackage{amsmath}

\usepackage{graphicx} %插入图片的宏包
\usepackage{float} %设置图片浮动位置的宏包
\usepackage{subfigure} %插入多图时用子图显示的宏包

%\usepackage{pdfpages}

\usepackage{geometry}
\geometry{left=3.17cm, right=3.17cm, top=2.54cm, bottom=2.54cm}   %页边距
\linespread{1}      %设置行距


% 摘要设置  -->设置为4号字
\renewcommand{\abstractname}{\textbf{\zihao{4}摘\quad 要}}

% 设置标题
\ctexset{
	section={
		format=\bfseries\zihao{4}\heiti\centering
	}
}

% 设置目录
\setcounter{tocdepth}{2} %设定目录深度为2,即只显示到二级标题为止

% 设置页码
\pagestyle{plain}

\begin{document}\songti\zihao{5}
	{\zihao{3}\heiti
	\title{基于R语言的网络数据获取和基本文本分析}
	\date{}
	\maketitle
	}
	

	\begin{abstract}
		随着大数据时代的到来,互联网上产生了越来越多的数据具有很大的参考价值,这些数据中往往蕴含着充分的政治、经济、社会人文等各种充满价值的信息,如果充分利用互联网信息来解读社会发展脉络,提炼其背后所蕴含的信息意义重大。本文基于R语言通过网络爬虫技术获得具有代表性的中文评论,然后通过jieba进行分词处理,对分词之后的中文文本进行瓷······
		
		
		\vspace*{1\baselineskip}   %空白间距
		%\centering   %可以把关键词放在中间
		
		\noindent{\textbf{关键词:}爬虫\quad 词频统计 \quad 绘制词云 }
	\end{abstract}

	\tableofcontents  %目录部分
	\newpage
	\section{爬虫}
	\subsection{获取URL}

% 关于\hfill 的用法	
%	A \hfill A \phantom{1} A \hfill A \\
%	A \hspace{\stretch{2}} A \hfill A \hfill A \\
	
	这是一个空白行
	\subsection{提取数据}
	
	这是一个空白行
	
	\section{获取想要的数据}

	这是一个空白行
	
	\section{数据处理}
	
	\subsection{去停用词}
	
	1.	题目自拟。
	
	2.	正文应包括以下基本内容:对项目原问题简要叙述、对数据的基本描述、相关统计概念、原理或方法、求解计算、结果分析、结论并做出必要的评价或评论等,将R语言代码放在附录里。正文字数2000~3000字。
	
	3. 论文第1页起开始编写页码,页码位于每页页脚中部,用阿拉伯数字从“1”开始连续编号。论文或报告题目用3号黑体字、一级标题用4号黑体字,并居中。报告中其他汉字一律采用5号黑色宋体字,行距使用单倍行距。
	
	4.	引用别人的成果或其他公开的资料(包括网上查到的资料) 必须按照规定的参考文献的表述方式在正文引用处和参考文献中均明确列出。正文引用处用方括号标示参考文献的编号,如[1][3]等;引用书籍还必须指出页码。参考文献按正文中的引用次序列出,
	其中书籍的表述方式为:  
	
	[编号] 作者,书名,出版地:出版社,出版年。
	
	参考文献中期刊杂志论文的表述方式为:
	
	[编号] 作者,论文名,杂志名,卷期号:起止页码,出版年。
	
	参考文献中网上资源的表述方式为:
	
	[编号] 作者,资源标题,网址,访问时间(年月 日)。\\
	5.	论文电子版提交时间:2021年6月23日(最后一次课)之前上传BB系统,文件名:学号-姓名-上课班级,比如1812345-张三-周三56节。用A4纸双面打印,不用装订(或左上角装订)。论文纸稿提交时间:2021年6月23日(最后一次课)之前。\\
	
	
	\subsection{去除停用词}

	
	这是一个空白行
	
	\subsection{载入停用词表}
	这是一个空白行
	
	\section{词频统计}
	这是一个空白行


	\begin{thebibliography}{99}    %参考文献开始
	\bibitem{1}失野健太郎.几何的有名定理.上海科学技术出版社,1986.      
	\bibitem{quanjing}曲安金.商高、赵爽与刘辉关于勾股定理的证明.数学传播,20(3),1998.  
	\bibitem{Kline}克莱因.古今数学思想.上海科学技术出版社,2002.
	\end{thebibliography}

	\addcontentsline{toc}{section}{参考文献}
	
	\begin{appendix}
		
		\section{附录}
		\small 勾股定理又叫商高定理,国外也称百牛定理。
		%\includepdf{Work.pdf}
		
	\end{appendix}
\end{document}